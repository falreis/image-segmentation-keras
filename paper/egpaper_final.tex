\documentclass[10pt,twocolumn,letterpaper]{article}

\usepackage{cvpr}
\usepackage{times}
\usepackage{epsfig}
\usepackage{graphicx}
\usepackage{amsmath}
\usepackage{amssymb}

\def\cvprPaperID{1} % *** Enter the CVPR Paper ID here

\usepackage[breaklinks=true,bookmarks=false]{hyperref}

\cvprfinalcopy % Comment this line and it stop working! :(
\ifcvprfinal\pagestyle{empty}\fi

\def\httilde{\mbox{\tt\raisebox{-.5ex}{\symbol{126}}}}

% Pages are numbered in submission mode, and unnumbered in camera-ready
%\ifcvprfinal\pagestyle{empty}\fi
\setcounter{page}{1}

\graphicspath{ {./images/} } 

\sloppy

%-------------------------------------------------------------------------
%-------------------------------------------------------------------------

\begin{document}

%%%%%%%%% TITLE
\title{Convolutional Neural Network to Image Segmentation}

\author{Felipe Augusto Lima Reis\\
PUC Minas - Pontif\'icia Universidade Cat\'olica de Minas Gerais\\
R. Walter Ianni 255 - Bloco L - Belo Horizonte, MG, Brasil\\
{\tt\small falreis@sga.pucminas.br}
}

\maketitle
%\thispagestyle{empty}

%%%%%%%%% ABSTRACT
\begin{abstract}
   The ABSTRACT is to be in fully-justified italicized text, at the top
   of the left-hand column, below the author and affiliation
   information. Use the word ``Abstract'' as the title, in 12-point
   Times, boldface type, centered relative to the column, initially
   capitalized. The abstract is to be in 10-point, single-spaced type.
   Leave two blank lines after the Abstract, then begin the main text.
   Look at previous CVPR abstracts to get a feel for style and length.
\end{abstract}

%%%%%%%%% BODY TEXT
\section{Introduction}

Image segmentation refers to the partition of an image into a set of regions to cover it, to represent meaningful areas \cite{DOMINGUEZ}. In this task, it's possible to take pixels as base unit in most image processing tasks\cite{WANG201728}.  Superpixels are the result of perceptual grouping of pixels, or seen the other way around, the results of an image oversegmentation \cite{WANG201728}. One example of oversegmentation can be seen in Figure \ref{fig:superpixel}. Superpixel can cause substantial speed-up of subsequent processing since the number of superpixels of an image reduces in contrast with the original number of pixels \cite{WANG201728}.

\begin{figure}[ht]
  \centering
  \includegraphics[width=0.48\textwidth]{superpixels.png}
  \caption{Segmented images using SLICO and EGB superpixels}
  \label{fig:superpixel}
\end{figure}

A utilização de superpixels possibilita a redução de itens a serem processados, entretanto pode causar perda de informação importante. No entanto, para alguns casos, a perda de qualidade pode se justificar em relação ao ganho de velocidade obtido utilizando esse tipo de operação. Essa relação consiste então em um \textit{trade-off} entre ambas as características, sendo viáveis em alguns cenários de processamento em tempo real ou para dispositivos com baixo desempenho.

Alguns métodos de geração de superpixels são utilizados para segmentação de imagens e detecção de bordas, como os métodos EGB \cite{FELZENSZWALB} e SLIC \cite{SLIC}. Esse trabalho investiga se a utilização de métodos segmentação baseados em superpixels, a composição de métodos de \textit{oversegmentation} e a utilização de redes neurais convolucionais para segmentação. 

Em relação aos métodos de segmentação utilizando redes neurais, o trabalho também tentará identificar se o treinamento utilizando imagens pré processadas obtêm resultados semelhantes àqueles utilizando imagens originais, na etapa de validação.

O presente trabalho apresenta a seguinte estrutura: a Seção \ref{sec:ref_teorico} mostra o referencial teórico para construção do trabalho, a Seção  , exibe os materiais e métodos utilizados nos testes; a Seção \ref{sec:testes} mostra os resultados obtidos nos testes realizados e a discussões dos mesmos; a Seção \ref{sec:conclusao} contém a conclusão do artigo, com as considerações finais.

The organization of this paper is as follows. In the next Section we discuss some related work and segmentations methods.  In Section \ref{sec:mat_metodos} its explained the method used in this paper. Then in Section \ref{sec:testes} we present an the results of the tests made for this paper and analise them. In Section 5 conclude the work.

%-------------------------------------------------------------------------
\section{Related Work} \label{sec:ref_teorico}

%-------------------------------------------------------------------------
\section{Segmentation Method} \label{sec:mat_metodos}

%-------------------------------------------------------------------------
\section{Tests and Results} \label{sec:testes}

%-------------------------------------------------------------------------
\section{Conslusion} \label{sec:conclusao}

%-------------------------------------------------------------------------

{\small
\bibliographystyle{ieee}
\bibliography{egbib}
}

\end{document}
